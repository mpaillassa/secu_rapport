\section{Que faire face aux failles Poodle et Freak}

\paragraph{}
Concernant les deux failles, la solution générale pour les deux et qui est aussi la plus radicale est la désactivation du support SSLv3 c\^oté client et seveur ( ce qui revient à faire une mise à jour des navigateurs). Une autre solution spécifique à chaque faille serait de désactiver le chiffrement qui pose problème :
\begin{itemize}
\item pour POODLE : les algorithmes utilisant le mode de chiffrement CBC défectueux
\item pour FREAK : RSA 512 bits
\end {itemize}

\paragraph{}
Les limites de la solution générale résident dans l'abandon total de SSLv3, ce qui induit qu'il y aura des problèmes de compatibilité avec les utilisateurs de SSLv3 (ceux sous Windows XP par exemple, puisque ce sytème n'a plus de support).

Ainsi, si tous les utilisateurs et les serveurs faisaient des mises à jour, les failles comme FREAK et POODLE seraient inexistants. Cependant, certains sites de ventes préfèrent garder la compatibilité avec SSLv3 pour ne pas perdre des clients.

%Nous avons regardé l'un des sites de ventes en ligne français "vente-privee.com" qui semble toujours utiliser SSlv3 : \textit{Ce processus est basé sur le protocole HTTPS (implémentation de SSLv3 ou TLSv1 sur HTTP), l'un des plus robustes aujourd'hui.} .

Sécurité ou tune ? telle est la question .... 




 