\section*{Introduction}

Les failles de sécurité informatique Poodle, pour "Padding Oracle On Downgraded Legacy Encryption", et Freak, pour "Factoring Attack on RSA-EXPORT Keys", sont deux failles qui ont été respectivement détectées début Octobre 2014 et début Mars 2015. Elles utilisent des vulnérabilités d'anciennes versions du protocole TLS (anciennement SSL). Il s'agit de protocoles de sécurisation des échanges sur Internet. On peut les reconnaitre si l'URL commence par "https://" ou si une image de cadenas s'affiche près de la barre d'adresse.\\ 
Le protocole SSL (Secure Sockets Layer) était développé par Netscape, puis, quand l'IETF a poursuivi le développement le nom a changé en TLS (Transport Layer Security). \\
Nous allons voir dans la suite quel sont leur fonctionnement, les risques qu'elles engendrent et comment y remédier. 
