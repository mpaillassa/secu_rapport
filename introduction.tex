\section*{Introduction}

Les failles de sécurité informatique Poodle, pour "Padding Oracle On Downgraded Legacy Encryption", et Freak, pour "Factoring Attack on RSA-EXPORT Keys", sont deux failles qui ont été respectivement officialisées début Octobre 2014 par Google et début Mars 2015 par l'INRIA et Microsoft Research. Elles utilisent des vulnérabilités d'anciennes versions du protocole TLS (SSL). Le protocole SSL (Secure Sockets Layer) était développé par Netscape, puis, quand l'IETF a poursuivi le développement le nom a changé en TLS (Transport Layer Security). 

 
Il s'agit de protocoles de sécurisation des échanges sur Internet. On peut les reconnaitre avec l'URL qui commence par "https://" ou avec une image de cadenas s'affichant près de la barre d'adresse.

 
Nous allons voir dans la suite quels sont le fonctionnement de ces failles, les risques qu'elles engendrent et comment y remédier. 
