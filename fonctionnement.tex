\section{Fonctionnement des failles}


Les deux failles utilisent le m\^eme principe et tiennent leur existence du fait que les anciennes versions des protocoles SSL/TLS comportent des failles. L'adversaire utilise une attaque de type "man in the middle" et force l'utilisation d'une version vulnérable de SSL, de manière à pouvoir exploiter les vulnérabilités qu'elle comporte. Parfois il n'est m\^eme pas utile de la forcer, car quand un navigateur doté d’une version récente n’arrive pas à se connecter à un serveur non mis à jour, ou mal configuré, il réessaie en utilisant des versions anciennes : c’est la downgrade dance, qui aboutit à l'utilisation d'une ancienne version vulnérable.


\subsection{Un point sur SSL/TLS}



\subsection{La faille Poodle}

\subsection{La faille Freak}


