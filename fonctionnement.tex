\section{Fonctionnement des failles}

\paragraph{}
Les deux failles POODLE et FREAK utilisent le m\^eme principe et tiennent leur existence du fait que les anciennes versions des protocoles SSL/TLS comportent des failles. L'adversaire utilise une attaque de type "man in the middle" et force l'utilisation d'une version vulnérable de SSL/TLS (en l'occurrence SSLv3), de manière à pouvoir exploiter ses failles. 

Parfois il n'est m\^eme pas utile de la forcer, car quand un navigateur doté d’une version récente n’arrive pas à se connecter à un serveur non mis à jour, ou mal configuré, il réessaie en utilisant des versions anciennes : c’est la downgrade dance, qui peut aboutir à l'utilisation d'une ancienne version vulnérable.

\subsection{Un point sur SSL/TLS}

\subsubsection{Historique}

\paragraph{}
SSL/TLS sont des protocoles qui fonctionnent suivant un mode client-serveur, et qui garantissent l'authentification du serveur et du client (optionnel), ainsi que la confidentialité et l'intégrité de données echangées. SSL est l'ancien nom de TLS, et a vécu sur 3 versions:
\begin{itemize}
\item 1994: SSL 1.0
\item Février 1995: SSL 2.0, première version réellement utilisée.
\item Novembre 1996: SSL 3.0 
\end{itemize}

\paragraph{}
TLS a ensuite pris le relais: 
\begin{itemize}
\item Janvier 1999: TLS 1.0
\item Avril 2006: TLS 1.1
\item Avril 2008: TLS 1.2
\end{itemize}

\paragraph{}
TLS, comme SSLv3, intègre un mécanisme de compatibilité ascendante avec les versions précédentes. Ainsi, quand le client et le serveur négocient la version du protocole qu'ils vont utiliser, par défaut, ils vont tenter de choisir la version la plus récente commune aux deux. 

\paragraph{}
Aujourd'hui la plupart des navigateurs supportent la version TLS 1.2 : 
\begin{itemize}
\item Google Chrome 30 et suivants
\item Mozilla Firefox 27 et suivants
\item Microsoft Internet Explorer 11 et suivants
\item Opera 17 et suivants
\item Apple Safari 7 et suivants
\end{itemize}

\paragraph{}
Ces navigateurs supportent la dernière version mais restent compatibles avec les anciennes aussi sauf la SSLv2 dont la compatibilité a été abandonné en Mars 2011. Cependant, la compatibilité avec SSLv3 persiste alors que cette dernière comporte plusieurs failles et ce sont celles-ci qui ont été exploité par les attaques POODLE et FREAK.  


\subsubsection{TLS Handshake protocol}

TLS est composé de deux niveaux: TLS Record Protocol (couche transport) et TLS Handshake Protocol (couche session), comme présenté sur la figure \ref{ssl-tls-diag}: \\

\begin{figure}[H]
  \centering
  \includegraphics[scale=0.6]{img/ssl-tls-diag.jpg}
  \caption{SSL/TLS layer}
  \label{ssl-tls-diag}
\end{figure}  


Ce protocole permet l'authentification et l'échange de clés nécessaires pour établir une session sécurisée. La figure \ref{handshake} décrit les différentes étapes de ce protocole:
\begin{figure}[H]
  \centering
  \includegraphics[scale=0.8]{img/tls-handshake.png}
  \caption{TLS handshake}
  \label{handshake}
\end{figure}  

Détails de chaque étape :\\


\textbf{ClientHello} : le client envoie ce message au serveur, pour lui indiquer ce qu'il supporte en terme de  :
\begin{itemize}
\item la dernière version TLS 
\item la liste de algorithmes de chiffrements  
\item les méthodes de compression
\end{itemize}
Il envoie également un nombre aléatoire.\\


\textbf{ServerHello} : le serveur envoie ce message au client, en indiquant ce qui sera utilisé en terme de : 
\begin{itemize}
\item la version TLS
\item l'algorithme de chiffrement
\item la méthode de compression
\end{itemize}
Il envoie également son certificat (permettant l'authentification) et un nombre aléatoire\\

\textbf{ServerHelloDone} : le client envoie ce message au serveur pour indiquer qu'il a fini avec la première phase du handshake.\\

\textbf{PreMasterSecret} : le client génère un nombre de ce nom, chiffré avec la clé publique du certificat du serveur.\\

\textbf{MasterSecret} : le client et le serveur génèrent alors leur secret partagé "Master Secret" à partir de leur nombre aléatoire et du "PreMasterSecret".\\

\textbf{ChangeCipherSpec} : le client envoie ce message en indiquant que la suite de la communication sera chiffrée, et envoie un message "Finished" qui est donc crypté.\\

\textbf{ChangedCipherSpec} : le serveur reçoit et déchiffre le message et envoie son "ChangedCipherSpec".  Le client reçoit ce dernier message, le déchiffre et c'est la fin du handshake.\\

Les algorithmes utilisés pour l'échange de clés sont DH (Diffie-Hellman) et RSA. 

\subsubsection{TLS Record Protocol}

\paragraph{}
Ce protocole permet de sécuriser les données en utilisant les clés qu'on a crée pendant le handshake. Il permet également de vérifier l'intégrité des données. La figure \ref{record} présente son fonctionnement:
\begin{figure}[H]
\centering
\includegraphics[scale=0.7]{img/tls-record.png}
\caption{TLS record}
\label{record}
\end{figure}

Détails de chaque étape:
\begin{itemize}
\item on divise le message à transmettre en fragments 
\item on compresse les fragments
\item on applique MAC (message authentification code) pour authentifier les messages 
\item on chiffre les messages
\item on ajoute un en-t\^ete
\end{itemize}

\paragraph{}
Quand on reçoit un message, on fait l'opération inverse (déchiffrement, vérification d'authenfication avec le MAC, décompression, défragmentation). 

Les algorithmes utilisés pour la phase de chiffrement sont par exemple RC4, DES, 3DES, AES. 

Le fonctionnement global décrit par les protocoles TLS handshake et TLS record est très similaire entre toutes les versions de SSL/TLS. 

\paragraph{}
La sécurité du protocole est donc gérée par deux éléments:
\begin{itemize}
\item un chiffrement asymétrique qui permet, après authentification de la clé publique du serveur, la constitution d'un secret partagé entre le client et le serveur. C'est ce qui a lieu au cours du handshake
\item un chiffrement symétrique qui est utilisé dans la phase d'échange de données, car beaucoup moins coûteux et plus rapide. C'est ce qui a lieu dans la phase de chiffrement du record. Les clés de chiffrement symétrique sont calculées à partir du secret partagé (de plus, une fonction de hachage est également utilisée pour assurer l'intégrité et l'authentification des données)
\end{itemize}

\paragraph{}
La faille Poodle exploite une faille de SSLv3 au niveau du mode de chiffrement symétrique CBC: le padding (complétion d'un bloc de bits pour respecter un format) permet de déchiffrer des messages. 


La faille Freak exploite une faiblesse de SSLv3 au niveau du chiffrement asymétrique RSA: de nos jours on peut casser une clé RSA de 512 bits facilement, et SSLv3 utilise des clés RSA de 512 bits.

\subsection{La faille Poodle}
\paragraph{}
Comme on le voit sur la figure \ref{poodle}, le principe repose sur une attaque de type "man in the middle" permettant de forcer l'utilisation de SSLv3 entre le client et le serveur dans le but d'exploiter ses failles. 

\begin{figure}[H]
\centering
\includegraphics[scale=0.9]{img/poodle.jpg}
\caption{fonctionnement de la faille poodle}
\label{poodle}
\end{figure}

Avec cette méthode, l'attaquant n’a pas accès aux fichiers stockés dans l’appareil ciblé, mais il est capable d'intercepter ses données de connexion, comme par exemple les cookies déposés dans le navigateur par un serveur lors d’une session. Ainsi, sans posséder le mot de passe du client, l'attaquant peut agir dans les comptes du client : envoyer des messages sur Twitter en se faisant passer pour elle, lire et effacer son courrier électronique, etc.

\subsubsection{Détails de la vulnérabilité de SSLv3 exploitée}

Deux méthodes de chiffrement symétrique sont principalement utilisées dans SSLv3:
\begin{itemize}
\item la première est RC4 (Rivest Cipher 4) qui est un chiffrement par flot (stream cipher). Le principe d'un chiffrement par flot est de générer un flot de bits aléatoires $k$ et de réaliser un XOR entre ce flot et le message à chiffrer $m$. L'avantage présenté par cette méthode réside en la rapidité du chiffrement et du déchiffrement : si $c = m \oplus k$, alors $m = c \oplus k$. 

Cependant, RC4 est actuellement considéré comme peu s\^ur et ne devrait plus \^etre utilisé
\item l'autre méthode est d'utiliser un algorithme comme DES ou AES avec un mode de chiffrement par blocs cha\^inés (CBC pour cipher block chaining), et c'est elle qui fait l'objet de l'attaque Poodle que l'on va détailler.
\end{itemize}

\paragraph{Fonctionnement du chiffrement par blocs cha\^inés}

Les chiffrements et déchiffrements par blocs cha\^inés sont illustrés en figure \ref{cbc-enc} et \ref{cbc-dec} (Plaintext est le message en clair noté $m$ et Ciphertext est le message chiffré noté $c$) \\

\noindent On utilise la formule suivante pour chiffrer: \\
$c_i = Enc_K(m_i \oplus c_{i-1}),\ c_0 = IV$

\begin{figure}[H]
\centering
\includegraphics[scale=0.6]{img/cbc-enc.png}
\caption{CBC encryption}
\label{cbc-enc}
\end{figure}

\noindent On utilise la formule suivante pour déchiffrer: \\
$m_i = Dec_K(c_i) \oplus c_{i-1},\ c_0 = IV$ 

\begin{figure}[H]
\centering
\includegraphics[scale=0.6]{img/cbc-dec.png}
\caption{CBC decryption}
\label{cbc-dec}
\end{figure}

Si un bit du message chiffré change, le bloc de message en clair correspondant est complètement corrompu et cela inverse le bit correspondant dans le bloc suivant de message en clair, mais le reste du bloc ne change pas. C'est ce qui est exploité par Poodle.

Supposons que l'attaquant ait trois blocs chiffrés $c_1$, $c_2$ et $c_3$ et qu'il souhaite déchiffrer le second bloc (et donc obtenir $m_2$). Il sait seulement que le dernier bloc est complété (padded) en utilisant la méthode PKCS7, c'est à dire que le dernier bloc est complété avec $n$ octets, chacun égal à $n$.

Le déchiffrement fonctionne comme ceci: $m_i = Dec(c_i) \oplus c_{i-1}$, avec $c_0 = IV$. Si l'attaquant change le dernier octet de $c_1$ et envoie ($IV$, $c_1$, $c_2$) au serveur, cela va affecter tout le bloc $m_1$ et le dernier octet de $m_2$. Ensuite le serveur va vérifier le dernier bloc déchiffré ($m_2$) et retourner oui ou non si la complétion ("padding") est correcte.

Si $b_{-1}$ est le dernier octet de $c_1$, l'attaquant le change comme suit: $b_{-1} = b_{-1} \oplus z_{-1} \oplus$ 0x01, où $z_{-1}$ est la valeur du dernier octet de $m_2$. Si $z_{-1}$ est correct, le serveur ne va pas trouver d'erreur de padding (parce que le dernier octet de $m_2$ sera égal à 0x01, ce qui est un padding correct). Dans l'autre cas, le serveur va trouver une erreur de padding et l'attaquant va essayer une autre valeur de $z_{-1}$. Dans le pire des cas, il doit faire 255 essais pour trouver la bonne valeur de $z_{-1}$.

Une fois qu'il conna\^it le dernier octet de $m_2$, l'attaquant peut obtenir l'avant-dernier octet de $m_2$. Il change les deux derniers octets de $c_1$: $b_{-1} = b_{-1} \oplus z_{-1} \oplus$ 0x02 et $b_{-2} = b_{-2} \oplus z_{-2} \oplus$ 0x02. Après 255 essais au maximum, il trouve $z_{-2}$ et ainsi de suite. 

Si chaque bloc fait 128 bits (AES), soit 16 octets, l'attaquant va obtenir $m_2$ en moins de 255x16 = 4080 essais. Cette attaque ne co\^ute presque rien et peut \^etre réalisée en quelques secondes.

La publication de Google qui a officialisé le faille offre les détails nécessaires pour récupérer les cookies d'une cible: This POODLE Bites: Exploiting The SSL 3.0 Fallback, Security Advisory, Bodo Möller, Thai Duong, Krzysztof Kotowicz, Google, September 2014.


\subsection{La faille Freak}
\paragraph{}
Le principe global de fonctionnement est le m\^eme que pour la faille POODLE (une attaque de type "man in the middle" qui force l'utilisation de SSLv3). Cependant, c'est encore une autre vulnérabilité de SSLv3 qui a été exploité dans FREAK.

\subsubsection{Détails de la vulnérabilité de SSLv3 exploitée}
\paragraph{}
Il s'agit ici de la taille de la clé employée est rpbolématique : si l'algorithme de chiffrement asymétrique utilisé pendant le handashake est RSA, alors sous SSLv3la clé est de 512 bits. Cette taille de 512 bits a été choisie par la NSA dans les années 1990, et était juste suffisante pour que eux seuls puissent casser ces clés. Seulement, de nos jours, casser une clé RSA de 512 bits est très abordable et pour cette raison les versions ultérieures  (TLS 1.0, 1.1 et 1.2) utilisent des clés plus grandes (1024/2048 bits). D'où l'inter\^et pour l'attaquant de forcer l'utilisation de SSLv3 par une attaque "man in the middle".

\paragraph{Fonctionnement du chiffrement RSA}

Le chiffrement RSA sert à établir un secret entre deux entités, par convention Alice (A) et Bob (B). Dans le cas du protocole SSL/TLS, on utilise le chiffrement RSA pour étalbir la clé qui sert au chiffrement symétrique (CBC, AES, DES) des communications. 

RSA est un chiffrement asymétrique, c'est à dire qu'il utilise deux clés: une publique pour chiffrer et une privée pour déchiffrer. Le principe étant que déchiffrer avec comme seule connaissance la clé publique soit calculatoirement impossible. \\

Le chiffrement RSA fonctionne en 3 étapes:
\begin{itemize}
\item \textbf{création des clés} \\
La création des clés se fait du c\^oté d'Alice par l'algorithme suivant:
\begin{itemize}
\item choisir $p$ et $q$, deux nombres premiers distincts
\item calculer le produit $n = pq$, appelé module de chiffrement 
\item calculer $\phi(n) = (p - 1)(q -1)$ (c'est la valeur de l'indicatrice d'Euler en $n$) 
\item choisir un entier naturel $e$ premier avec $\phi(n)$ et strictement inférieur à  $\phi(n)$, appelé exposant de chiffrement
\item calculer l'entier naturel $d$, inverse de $e$ modulo  $\phi(n)$, et strictement inférieur à  $\phi(n)$, appelé exposant de déchiffrement ($d$ peut se calculer efficacement par l'algorithme d'Euclide étendu)
\end{itemize}
Comme $e$ est premier avec  $\phi(n)$, d'après le théorème de Bachet-Bézout il existe deux entiers $d$ et $k$ tels que $ed + k \phi(n) = 1$, c'est-à-dire que $ed = 1\ mod\ \phi(n)$ : $e$ est bien inversible modulo $\phi(n)$.

Le couple ($n$,$e$) est la clé publique du chiffrement, alors que le couple ($n$,$d$) est sa clé privée.

\item \textbf{chiffrement du message} \\
Si $M$ est un entier naturel strictement inférieur à $n$ représentant un message, alors le message chiffré sera représenté par: \\
$C = M^e\ mod\ n$ \\
L'entier naturel $C$ étant choisi strictement inférieur à $n$.

\item \textbf{déchiffrement du message} \\
Pour déchiffrer $C$, on utilise $d$, l'inverse de $e\ mod\ (p-1)(q-1)$, et on retrouve le message clair $M$ par: \\
$M = C^d\ mod\ n$ \\

\end{itemize}

Pour chiffrer un message, il suffit de connaître $e$ et $n$ alors que pour déchiffrer, il faut $d$ et $n$. 

Pour calculer $d$ à l'aide de $e$ et $n$, il faut trouver l'inverse modulaire de $e$ modulo $(p - 1)(q - 1)$ ce que l'on ne sait pas faire sans connaître les entiers $p$ et $q$, c'est-à-dire la décomposition de $n$ en facteurs premiers. 

Une attaque par force brute consisterait donc à retrouver $p$ et $q$ en ne connaissant que $n$. Factoriser un entier $n$ en $pq$ avec $p$ et $q$ premiers est reconnu comme étant un problème très difficile (bien que ce ne soit pas prouvé) et dans le pire des cas, sur une clé de 512 bits, on peut tester tous les nombres premiers inférieurs à $\sqrt{2^{512}} = 2^{256}$, et il y en a environ $\frac{2^{256}}{256 \times \ln(2)} = 6.5 \times 10^{74}$ (en utilisant la formule $\pi(x) \approx \frac{x}{\ln(x)}$, où $\pi(x)$ donne le nombre d'entiers premiers avec $x$ et inférieurs à $x$).

En 2009, Benjamin Moody a cassé une clé RSA de 512 bits en 73 jours en utilisant seulement un logiciel public (GGNFS) et son ordinateur personnel (dual-core Athlon64 à 1900 MHz): 5 Go d'espace disque et 2.5 Go de RAM lui ont été nécessaires. \\

Pour donner un ordre d'idée, voici la factorisation faite par Benjamin Moody : 

$5174413344875007990519123187618500139954995264909695897020209972309881454541 $\\ 
$ \times  1325290319363741258636842042448323483211759628292406959481461131759210884908747 $ \\
$ =  68575999143494039776547449671727581799041142646129473261271699761332969809514505427$\\
\indent $89808884504301075550786464802304019795402754670660318614966266413770127$


%La sécurité de l'algorithme RSA contre les attaques par la force brute repose donc sur le fait que casser RSA de cette manière nécessite la factorisation du nombre $n$ en le produit initial des nombres $p$ et $q$, et qu'avec les algorithmes classiques, le temps que prend cette factorisation croît exponentiellement avec la longueur de la clé. \\




